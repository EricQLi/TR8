\documentclass{article}
\usepackage{nameref}
\usepackage{url}
\bibliographystyle{plain}
%% \VignetteIndexEntry{Plants traits data}
%%\VignetteDepends{TR8}
\title{TR8: Extract traits data for plant species}
\author{Gionata Bocci\\Pisa (ITALY)\\ {boccigionata@gmail.com}}
\usepackage{Sweave}
\begin{document}
\maketitle



\section{Rationale}
\label{sec:rationale}

The \texttt{TR8} package has been built in order to provide the user with the
possibility of easily retrieving traits data for plant species from the following publicly available databases:

\begin{description}
\item[Biolflor] \url{http://www2.ufz.de/biolflor/index.jsp} \cite{biolflor}
\item[Ecological Flora of the British Isles] \url{http://www.ecoflora.co.uk/} \cite{ecoflora}
\item[LEDA traitbase] \url{http://www.leda-traitbase.org/LEDAportal/} \cite{leda}
\item[Ellenberg values for Italian Flora] \cite{pignatti}
\item[Mycorrhizal intensity database] \cite{amf}
\end{description}

  Please note that not all the traits available on the listed
  databases are downloaded by the package: this may change in future
  versions of the package (ie. some functionalities may be added and
  more traits will be made available).

\section{Installation}
\label{sec:installation}

 The package relies on some functions provided by the following packages (which are thus needed):
 \begin{itemize}
  \item plyr\cite{plyr}
  \item reshape\cite{reshape}
  \item RCurl\cite{RCurl}
  \item XML\cite{XML}
  \item taxize\cite{taxize}
  \item gWidgets\cite{gWidgets}
  \end{itemize}
  
   For the moment the package is available only as a \texttt{.zip}
   file, thus you will have to manually install these other packages; to do so, run the following:

\begin{Schunk}
\begin{Sinput}
> install.packages(c("XML","RCurl","plyr","taxize","methods","gWidgets"), dependencies=TRUE)
\end{Sinput}
\end{Schunk}

   
  In order to install the \texttt{TR8} package from the 
  \texttt{R commander} (\texttt{Rcmdr}) GUI interface,  go to the \texttt{Packages} menu,
  then choose \texttt{Install package(s) from local zip files} and choose the
  provided compressed package file.
  
  Once the package is installed, you can load it with:

\begin{Schunk}
\begin{Sinput}
> library(TR8)
\end{Sinput}
\end{Schunk}

Please note that:

\begin{description}
\item[The user is asked to always cite the data sources: ] the
  development of traits databases is a long and costly process,
  thus all the users of the \texttt{TR8} package are asked (and
  reminded \textbf{every time} they load the package and use the \texttt{tr8()} function) to always cite the original sources of the data (see
  paragraph \ref{sec:citing}).
  
\end{description}
  
\section{Simple usage}
\label{sec:usage}

  Using the \texttt{TR8} package is fairly simple: users just need to
  call the \texttt{tr8} function passing, as argument, a vector of
  plant species names:
  
\begin{Schunk}
\begin{Sinput}
> ## a vector containing a list of plant species names
> my_species<-c("Apium graveolens","Holcus mollis","Lathyrus sylvestris")
> ## now run tr8 and store the results in the my_traits object
> my_traits<-tr8(my_species)
\end{Sinput}
\end{Schunk}



  A multi-panel window will appear: the user is asked to choose those
  traits which are to be downloaded from the various databases; if data from the LEDA Traitbase are selected, after
  clicking \texttt{OK} a second GUI will appear, asking if a local copy
  of the \texttt{LEDA} \texttt{.txt} files has already been downloaded
  (see paragraph \ref{sec:leda} for a more detailed explanation of the
  way \texttt{TR8} deals with the LEDA Traitbase): if this is the case, please tick the
  "yes" button, click "Ok" and then choose the folder which contains
  the already downloaded files.  
  The \texttt{tr8} function will take care of downloading the data and
  store them in the \texttt{my\_traits} object; you can see the results
  by printing them:

\begin{Schunk}
\begin{Sinput}
> ## see the downloaded data
> print(my_traits)
\end{Sinput}
\end{Schunk}

  Or you can convert them to a data frame using the \texttt{extract\_traits} function:
  
\begin{Schunk}
\begin{Sinput}
> traits_dataframe<-extract_traits(my_traits)
\end{Sinput}
\end{Schunk}



  All the traits are now contained in a data frame with species as rows
  and columns as traits; where no trait data were available, you will
  see a \texttt{NA}. 
  
  %% In order to let the user be able to choose which traits should be
  %% downloaded, the \texttt{tr8\_config()} function multi-panel
  %% \textit{GUI} : this selection of traits should be
  %% run \textbf{before}  
  %% using the \texttt{tr8()} function.
  
%%  @ 
%% <<usage,eval=FALSE>>=
%% ## run the tr8_config() function
%% tr8_config()
%% ## select the traits of interest and then close the window 

%% ## run tr8() 
%% my_species<-c("Apium graveolens","Holcus mollis","Lathyrus sylvestris")
%% my_traits<-tr8(my_species)
%% ## see the downloaded data
%% print(my_traits)
%%  @ %def 
  
  
  In order to make the dataframe more readable,  traits' names (ie. columns' names) are converted to shorter codes: to see a brief explanation of the codes used to identify the traits, use the \texttt{lookup} function: 


\begin{Schunk}
\begin{Sinput}
> lookup(my_traits)
\end{Sinput}
\end{Schunk}

  For a detailed explanation of each level of a trait, please refer to
  the original websites (all the databases listed in the references
  provide the users with very precise and detailed descriptions).


   Tipically users will have a their vegetation data in the form of a
   \textit{sites}*\textit{species} dataframe (or matrix), thus they
   may want to extract traits data for the whole dataset, ie.:
   
\begin{Schunk}
\begin{Sinput}
> ## suppose veg_data is our dataframe with
> ## plant species as columns and sites as rows
> 
> ## extract species names
> specie_names<-names(veg_data)
> ## use the tr8() function
> my_traits<-tr8(species_names)
> ## print the results
> print(my_traits)
\end{Sinput}
\end{Schunk}
  
  \textbf{A NOTE OF CAUTION}: searching the web is a time (and
  internet band) consuming activity, thus the higher the number of
  your plant species and the traits to be retrieved, the longer it will take to \texttt{tr8()} to complete its job. Moreover, in order
  not to overflow the remote databases with \texttt{http} requests, we
  suggest to split the the list of plant species in smaller chunks.

  \textbf{A (SECOND) NOTE OF CAUTION}: some users adopt the following workflow for analysing their vegetation data:

  \begin{enumerate}
  \item insert vegetation data into a \textit{spreadsheet file} with species as
  columns' and sites' as rows
\item export the spreadsheet file as a \texttt{.csv} file
\item import the \texttt{.csv} file into a \textbf{R} dataframe.
  \end{enumerate}
  
  When following these steps, a dot (".") will be inserted between
Genus and Species of each plant species name (i.e. column names in the
\texttt{R} dataframe will not be in the form \texttt{c("Abies alba", "Salix
alba")} but in the form  \texttt{c("Abies.alba", "Salix.alba")}).
This may cause problems for further processing of plants'
species names, thus, in order to avoid this problem, please use the \texttt{check.names=F}
  option in \texttt{read.csv}. Eg. suppose that
  \texttt{my\_veg\_data.csv} is the \texttt{csv} file: in the
  \textbf{R} console, one should use:

\begin{Schunk}
\begin{Sinput}
> My_data<-read.csv("my_veg_data.csv",header=T,row.names=1,check.names=F)
\end{Sinput}
\end{Schunk}


\section{Interpreting retrieved data}
\label{sec:interpreting}

Please note that for many traits there is more than one entry in the
original databases: in those cases, in order to obtain a single value
the following strategy was adopted:

\begin{description}
\item[Quantitative traits] the mean of all the values was calculated
  (eg. when multiple values for "Seed weight mean" are available, the
  mean of these value is calculated)
\item[Qualitative traits] all the values are taken into account and
  "joined" together in a single string (the values are separated by a
  score "$-$")
\end{description}
\section{Citing sources of information}
\label{sec:citing}

  Users of the \texttt{TR8} package should always cite the sources of information which provided the traits data: the correct citations to be used for the retrieved data can be obtained through the \texttt{bib} method; just use:
  

\begin{Schunk}
\begin{Sinput}
> bib(my_traits)
\end{Sinput}
\end{Schunk}

  
  
 \section{Suggested usage}

  We strongly suggest to always check plant species names with the
  \texttt{tnrs} function (from the \texttt{taxize} package) before
  using the \texttt{tr8} function; thus a typical workflow would be
  the following:
  
  \begin{enumerate}
  \item Check plant species names (eg. with something like the following -  please refere to the \texttt{taxize} package documentation\cite{taxize} for further details)
    
\begin{Schunk}
\begin{Sinput}
> species_names<-names(veg_data)
> checked_names<-tnrs(species_names,source="iPlant_TNRS")
> print(checked_names)
\end{Sinput}
\end{Schunk}

 Check which species (rows) in the table have a "score" value
 lower than 1 and check their names; if needed, correct them
 before using the tr8() function
\item Run \texttt{tr8}

\begin{Schunk}
\begin{Sinput}
> my_traits<-tr8(species_names)
> print(my_traits)
\end{Sinput}
\end{Schunk}
 
\item You may want to have these traits available as a data frame:
  just use the \texttt{extract\_traits} function which uses the results
  of \texttt{tr8} (in this case it's the \texttt{my\_traits} objects)
  and returns a data frame.

\begin{Schunk}
\begin{Sinput}
> traits_df<-extract_traits(my_traits)
\end{Sinput}
\end{Schunk}

\item Observing a big data frame inside \texttt{R} could be difficult,
  thus users may want to save the \texttt{traits\_df} data frame as a
  \texttt{.csv} file and open that with a spreadsheet software (eg. LibreOffice)
 
  \end{enumerate}
  

  



\section{LEDA Traitbase}
\label{sec:leda}

  The LEDA Traitbase datafiles are \texttt{.txt} files which are
  available for download at the LEDA website
  (\url{http://www.leda-traitbase.org/LEDAportal/data_files.jsp}). These
  files are (quite) big in size, thus downloading them every time the
  \texttt{tr8()} function is used is a time consuming
  activity\footnote{The .txt files are not distributed together with the
    \texttt{TR8} package - which would save time and memory when executing
    the \texttt{tr8()} function - in order to avoid possible licensing
    conflicts between the \texttt{TR8}' license and LEDA datasets.}. We
  thus suggest the users to run the
  \texttt{leda\_download\_to\_local\_directory()}\footnote{The name is quite
    self$-$explanatory...} function once to store a local copy of the LEDA
  datafiles and use that local copy every time the \texttt{tr8()} function is run.
  
  
\begin{Schunk}
\begin{Sinput}
> ## run the function
> leda_download_to_local_directory()
> ## and select a folder where the datafiles will be
> ## downloaded (remeber this folder, since you will be later
> ## ask for that when running the tr8() function)
\end{Sinput}
\end{Schunk}

  

  


 
\bibliography{tr8}



\end{document}
